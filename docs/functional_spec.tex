\documentclass{czen}
\author{Alan Chandler}
\title{Functional Specification}
\project{Football Mobile}
\usepackage{hyperref}
\begin{document}
\maketitle
\abstract{%
This is the functional specification for a project which takes the existing MBBall application and re-implements it using Google's Polymer Components.  This includes a responsive design so that it will be suitable for mobile devices as well as a full desktop, and Google's paper elements, that provide a look and feel for the appropriate components.
}
\tableofcontents
\section{Overall Scope}
\subsection{Introduction}
MBBall is a web based application used my Melinda's Backups (the fan club for Melinda Doolittle) on an annual basis to run a competition for picking the results each week of the NFL American Football matches.  This application was originally developed to replace a system of spreadsheets, but allows users to edit their own picks.  The matches, and the participants picks are stored in a database.  Initially this was a Postgres database, but was subsequently implemented using sqlite.  It is also itimately connected with the fan club's SMF forum.  Users are only allowed access to this application if they are a member of the forum, and membership of SMF groups controls who may have access rights to administrate the application.  The fan club also uses these Membership groups to indicate who are not adults (we call these members Baby Backups), and the application makes a distinction requiring approval for this class of member when they register to be a player in a competition

The current implementation is purely focused on users with a full sized screen.  User input, and results are all arranged in (largish) tables where substantial screen real estate is needed to see what is being presented.  The purpose of football-mobile is to re-implement the entire application so that it may be accessed from the wider range of devices available today, from a smart phone through tablets to the full size screen of laptop or desktop computer.

There is a secondary purpose in re-implementing this application, and that is to explore the possibilities and limitations of the new \emph{web components} aspect of web development, using Google's Polymer library as main engine.  Of particular interest is the use of test driven development, and how test scripts can interact with Polymer elements in order to comprehensively unit test the application.  Part of what Google offer with these Polymer based elements is a set of standard components for pages, and dialog boxes matchine their overall UI concept.  These are called \emph{Paper Elements} and will be used within the application. The other technological innovation is the use of Javascript throughout, including the server elements of the application\footnote{There is one exception to that, and that is the single script used to authenticate users, and get their user information, from the forum. This has to be a php script}.  Nodejs, and more specifically Expressjs web server application will be used as the host server\footnote{In production this may be sitting behind an Nginx front end.}.
\subsection{Concepts}
Since the existing application has developed over a number of years, the structure of the application, and the entities within it are relatively well known.  What will change is the user flow through the application as the limitations of the potentially smaller screens are worked around.  Given that the underlying concepts will not change, it makes sense to describe them here to ensure that the reader clear about the terms that are used and what they mean.  Rather than provide a long list, we group them into classes which are listed in the subsection headings below.
\subsubsection{The types of user for this application}
As listed in the introduction, the users for this application are intimately connected to the SMF forum that is the core of Melinda's Backups web site (at http://melindasbacksups.com).  This is used to identify (at least initially) classes of user.  However as also indicated above users have to actively sign up to participate in a competition\footnote{The term \emph{competition} is defined later in this section.}.  So lets describe what we have
\begin{description}
\item{Guest}This is a user who is either not logged in to the forum, or who is not even a member.  We cannot differenciate between the two.  Either way they are not allowed access to this application, so before doing anything, the application has to ask to forum to authenticate the user (done by checking whether they have a cookie presented to the forum).
\item{UID} this is a number, greater than zero, that represents the user ID of a member of the forum.  This is the \emph{primary key} of the user and can be used to unqiquely identify him/her.
\item{User Name}.  The name by which a user will be presented to other users on the user interface
\end{description}
\end{document}
